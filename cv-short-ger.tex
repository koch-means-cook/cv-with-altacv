%%%%%%%%%%%%%%%%%
% This is an sample CV template created using altacv.cls
% (v1.7, 9 August 2023) written by LianTze Lim (liantze@gmail.com). Compiles with pdfLaTeX, XeLaTeX and LuaLaTeX.
%
%% It may be distributed and/or modified under the
%% conditions of the LaTeX Project Public License, either version 1.3
%% of this license or (at your option) any later version.
%% The latest version of this license is in
%%    http://www.latex-project.org/lppl.txt
%% and version 1.3 or later is part of all distributions of LaTeX
%% version 2003/12/01 or later.
%%%%%%%%%%%%%%%%

%% Use the "normalphoto" option if you want a normal photo instead of cropped to a circle
% \documentclass[10pt,a4paper,normalphoto]{altacv}

\documentclass[10pt,a4paper,ragged2e,withhyper]{altacv}
%% AltaCV uses the fontawesome5 and packages.
%% See http://texdoc.net/pkg/fontawesome5 for full list of symbols.

% Change the page layout if you need to
\geometry{left=1.25cm,right=1.25cm,top=1.5cm,bottom=1.5cm,columnsep=1.2cm}

% The paracol package lets you typeset columns of text in parallel
\usepackage{paracol}

% Package to frame content
\usepackage{framed}

% Change the font if you want to, depending on whether
% you're using pdflatex or xelatex/lualatex
% WHEN COMPILING WITH XELATEX PLEASE USE
% xelatex -shell-escape -output-driver="xdvipdfmx -z 0" sample.tex
\ifxetexorluatex
  % If using xelatex or lualatex:
  \setmainfont{Roboto Slab}
  \setsansfont{Lato}
  \renewcommand{\familydefault}{\sfdefault}
\else
  % If using pdflatex:
  \usepackage[rm]{roboto}
  \usepackage[defaultsans]{lato}
  % \usepackage{sourcesanspro}
  \renewcommand{\familydefault}{\sfdefault}
\fi

% Package for multiple columns withi paracol (cannot be nested)
\usepackage{multicol}
% Package for nice rules between columns (see REFERENCE section)
\usepackage{multicolrule}
\usepackage{tikz}

% Change the colours if you want to
\definecolor{SlateGrey}{HTML}{2E2E2E}
\definecolor{LightGrey}{HTML}{666666}
%\definecolor{DarkPastelRed}{HTML}{450808}
%\definecolor{PastelRed}{HTML}{8F0D0D}
\definecolor{GoldenEarth}{HTML}{E7D192}
\colorlet{name}{black}
\colorlet{tagline}{black}
\colorlet{heading}{black}
\colorlet{headingrule}{GoldenEarth}
%\colorlet{subheading}{PastelRed}
\colorlet{subheading}{black}
%\colorlet{accent}{PastelRed}
\colorlet{accent}{black}
\colorlet{emphasis}{SlateGrey}
\colorlet{body}{LightGrey}

% Change some fonts, if necessary
\renewcommand{\namefont}{\Huge\rmfamily\bfseries}
\renewcommand{\personalinfofont}{\footnotesize}
\renewcommand{\cvsectionfont}{\LARGE\rmfamily\bfseries}
\renewcommand{\cvsubsectionfont}{\large\bfseries}


% Change the bullets for itemize and rating marker
% for \cvskill if you want to
\renewcommand{\cvItemMarker}{{\small\textbullet}}
\renewcommand{\cvRatingMarker}{\faCircle}
% ...and the markers for the date/location for \cvevent
\renewcommand{\cvDateMarker}{\faCalendarCheck[regular]}
\renewcommand{\cvLocationMarker}{\faMapMarker*}


% If your CV/résumé is in a language other than English,
% then you probably want to change these so that when you
% copy-paste from the PDF or run pdftotext, the location
% and date marker icons for \cvevent will paste as correct
% translations. For example Spanish:
% \renewcommand{\locationname}{Ubicación}
% \renewcommand{\datename}{Fecha}


%% Use (and optionally edit if necessary) this .tex if you
%% want to use an author-year reference style like APA(6)
%% for your publication list
% \input{pubs-authoryear.tex}

%% Use (and optionally edit if necessary) this .tex if you
%% want an originally numerical reference style like IEEE
%% for your publication list
%\input{pubs-num.tex}

%% sample.bib contains your publications
%\addbibresource{sample.bib}

\begin{document}
\name{Christoph Koch\textnormal{\huge,~D\MakeLowercase{r. rer. nat}}}
\tagline{Research Data Scientist mit Hintergrund in kognitiven Neurowissenschaften}
%% You can add multiple photos on the left or right
% \photoR{2.8cm}{photo-wittkuhn-uhh}
% \photoL{2.5cm}{Yacht_High,Suitcase_High}
\personalinfo{%
  % Not all of these are required!
  \email{christoph.koch1992@gmail.com}
  \phone{+49 1578 251 9078}
  %\mailaddress{Von-Melle-Park 5, D-20146 Hamburg, Germany}
  \location{Berlin, Germany}
  %\homepage{www.lennartwittkuhn.com}
  \github{koch-means-cook}
  \linkedin{dr-christoph-koch}
  %\twitter{@koch_means_cook}
  %\printinfo{\faMastodon}{@koch_means_cook@fediscience}[https://fediscience.org/@koch_means_cook]
  %\orcid{0000-0003-2966-6888}
  % \printinfo{\faGraduationCap}{Google Scholar}[https://scholar.google.com/citations?user=GXvJB1kAAAAJ&hl=en]
  %% You can add your own arbitrary detail with
  %% \printinfo{symbol}{detail}[optional hyperlink prefix]
  % \printinfo{\faPaw}{Hey ho!}[https://example.com/]

  %% Or you can declare your own field with
  %% \NewInfoFiled{fieldname}{symbol}[optional hyperlink prefix] and use it:
  % \NewInfoField{gitlab}{\faGitlab}[https://gitlab.com/]
  % \gitlab{your_id}
  %%
  %% For services and platforms like Mastodon where there isn't a
  %% straightforward relation between the user ID/nickname and the hyperlink,
  %% you can use \printinfo directly e.g.
  % \printinfo{\faMastodon}{@username@instace}[https://instance.url/@username]
  %% But if you absolutely want to create new dedicated info fields for
  %% such platforms, then use \NewInfoField* with a star:
  % \NewInfoField*{mastodon}{\faMastodon}
  %% then you can use \mastodon, with TWO arguments where the 2nd argument is
  %% the full hyperlink.
  % \mastodon{@username@instance}{https://instance.url/@username}
}

\makecvheader{}
%% Depending on your tastes, you may want to make fonts of itemize environments slightly smaller
% \AtBeginEnvironment{itemize}{\small}

%% Set the left/right column width ratio to 6:4.
\columnratio{0.6}

% Start a 2-column paracol. Both the left and right columns will automatically
% break across pages if things get too long.
\begin{paracol}{2}

  \cvsection{Über mich}
  %\cvachievement{\faChartArea}{Research Data Scientist}{5+ years experience in data analysis \& management, advanced statistics, machine learning \& data visualization}
  \cvachievement{\faChartArea}{Research Data Scientist}{5+ Jahre Erfahrung in Prozessierung/Analyse/Visualisierung \& Management von Daten, Statistik \& Machine Learning}
  \cvachievement{\faLaptopCode}{Flexibles und sich erweiterndes Toolkit}{Mehrere Programmiersprachen, Data Science Libraries, Versionskontrolle mit Git \& High Performance Computing (HPC)}
  \cvachievement{\faUsers}{Eigenständigkeit und Teamwork in Projektarbeit}{Hauptverantwortung für 3 Langzeitprojekte mit geringer Supervision \& mehrere Expertenrollen in internationalen Teams}
  \cvachievement{\faCubes}{Erfahrung mit komplexen, herausfordernden Datensätzen}{Inkl. monetäre Entscheidungen und Lernprozesse mehrerer demografischer Gruppen, 4D-Bildverarbeitung von Gehirnscans \& menschliche Bewegung durch virtuelle Räume}
  %\faStreetView
  %\faUserFriends
  %\faUsers
  %\faWalking
  %\cvachievement{\faBrain}{Cognitive Neuroscientist \& Psychologist}{Comprehensive understanding of human behavior \& cognition, incl.~motivation, decision-making \& memory processes} 
  %\cvachievement{\faUserGraduate}{Researcher \& Project Leader}{High-impact publications in top-tier scientific journals with \textasciitilde{}100 citations, project management, funding acquisition}
  %\cvachievement{\faChalkboardTeacher}{Speaker \& Teacher}{20+ presentations, talks \& workshops with 1000+ attendees at international conferences \& renowned research institutions}
  %\cvachievement{\faSlideshare}{Supervisor \& Consultant}{Supervised aspiring researchers at varying career levels \& from diverse technical backgrounds}

 \medskip
  
  \cvsection{Erfahrung}
  \cvevent{Postdoctoral Research Scientist}{Universität Hamburg\\Max-Planck-Institut für Bildungsforschung, Berlin}{02/2023 -- jetzt}{Hamburg \& Berlin}
  \begin{itemize}[itemsep=4pt]
    \item Nutzen von \textbf{Reinforcement Learning Algorithmen} zum Modellieren menschlicher Entscheidungen
    \item Fortgeschrittene Statistik \& Datenvisualisierung in \textbf{R}
    \item \textbf{Data management lead} in internationalem Projekt (6 Mitglieder)
    \item \textbf{Dokumentation, Kommunikation \& Vermittlung} von Inhalten durch Präsentationen, Workshops, Wikis \& wiss. Publikationen
    %\item \textbf{Research:} Multivariate pattern analysis of brain imaging data, computational modeling \& analysis of human behavior \& cognition to investigate fast memory reactivation in humans
    %\item \textbf{Skills:} Pattern classification, machine learning \& data analysis in Python (e.g., scikit-learn, pandas, numpy), computational modeling, advanced statistics \& data visualization in R, version-controlled code \& data management using Git \& DataLad, high performance computing (HPC), Docker, CI/CD on GitHub \& GitLab
    % Funding Acquisition
    % Git Workshop

    %\item \textbf{Teaching:} Full-semester \href{https://lennartwittkuhn.com/version-control-course-uhh-ws23/}{course on ``Version control of code and data using Git and DataLad''} for MSc Psychology students
    %\item \textbf{Lab Management:} Project Administration, , Supervision of Research Assistants,
  \end{itemize}
  
  \divider{}
  
  \cvevent{Doctoral Researcher}{Max-Planck-Institut für Bildungsforschung, Berlin}{11/2017 -- 12/2022}{Berlin, Germany}
  \begin{itemize}[itemsep=4pt]
    \item Machine learning, pattern classification \& Datenanalyse in \textbf{Python}
     \item Fortgeschrittene Statistik \& Datenvisualisierung in \textbf{R}
    \item \textbf{Selbstgeführtes Management von Langzeitprojekten} für Dissertation: \href{http://dx.doi.org/10.17169/refubium-39372}{``How aging shapes neural representations of continuous spaces''}
    \item Mitglied der \href{https://www.imprs-life.mpg.de/}{\textbf{International Max Planck Research School LIFE}} (Deutschland, USA \& Schweiz) + ein Jahr als Mitgliedssprecher
  \end{itemize}

  %\divider{}

  %\textbf{Previously:} Research and Teaching Assistant (Ruhr-University, Bochum, Germany)

% \cvsection{A Day of My Life}

% Adapted from @Jake's answer from http://tex.stackexchange.com/a/82729/226
% \wheelchart{outer radius}{inner radius}{
% comma-separated list of value/text width/color/detail}
% \wheelchart{1.5cm}{0.5cm}{%
%   6/8em/accent!30/{Sleep,\\beautiful sleep},
%   3/8em/accent!40/Hopeful novelist by night,
%   8/8em/accent!60/Daytime job,
%   2/10em/accent/Sports and relaxation,
%   5/6em/accent!20/Spending time with family
% }


\bigskip

%% Switch to the right column. This will now automatically move to the second
%% page if the content is too long.
\switchcolumn{}

\cvsection{Skills}

% Optimization using nlopt

\vspace*{-0.25em}
\bfseries\textcolor{emphasis}{Programming Languages}
\vspace*{-1.2em}
\begin{multicols}{2}
\textbf{\underline{Fortgeschritten:}}
\vspace{0.3em}
\begin{itemize}
    \item \textnormal{Python}
    \item \textnormal{R}
\end{itemize}
\vfill
\columnbreak
\textbf{\underline{Basic:}}
\vspace{0.3em}
\begin{itemize}
    \item \textnormal{Bash}
    \item \textnormal{Julia}
\end{itemize}
\end{multicols}
%\cvtag{\href{https://www.python.org/}{\faPython{}Python}}
%\cvtag{\href{https://www.r-project.org/}{\faRProject{}}}
%\cvtag{\href{https://en.wikipedia.org/wiki/Bash_(Unix_shell)}{\faTerminal{}Bash}}
%\cvtag{\href{https://www.mathworks.com/products/matlab.html}{\faMountain{}MATLAB}}
\vspace*{-1.5em}
\divider{}

\bfseries\textcolor{emphasis}{Tools}\\
\vspace{0.8em}
%\cvtag{\href{https://git-scm.com/}{\faGit*{}Git}}
%\cvtag{\href{https://www.docker.com/}{\faDocker{}Docker}}
%\cvtag{\href{https://about.gitlab.com/topics/ci-cd/}{\faCodeBranch{}CI/CD}}\\
%\cvtag{\href{https://scikit-learn.org/stable/}{\faLayerGroup{}Machine Learning}}
%\cvtag{\href{https://jupyter.org/}{\faRocket{}Jupyter}}\\
%\cvtag{\href{https://en.wikipedia.org/wiki/High-performance_computing}{\faServer{}HPC}}
%\cvtag{\href{https://www.datalad.org}{\faDatabase{}DataLad}}
%\cvtag{\href{https://aws.amazon.com/s3/}{\faAmazon{}S3}}\\
%\cvtag{\href{https://www.markdownguide.org/}{\faMarkdown{}Markdown}}
%\cvtag{\href{https://www.latex-project.org/}{\LaTeX{}}}
%\cvtag{\href{https://www.latex-project.org/}{\faMicrosoft{} MS Office}}
\begin{itemize}[label={}, left=0pt, itemsep=5pt, topsep=0pt]
    \item \faLayerGroup{}\ Machine Learning \textnormal{(scikit-learn)}
    \item \faBullseye{}\ Optimierung \textnormal{(NLopt, nloptr \& optimx)}
    \item \faBoxes{}\ Data Science Libraries \textnormal{(u.a.)}
    \begin{itemize}[left=7pt, itemsep=2pt, topsep=2pt]
        \item Python: \textnormal{scikit-learn, pandas, numpy}
        \item R: \textnormal{data.table, tidyr, ggplot2}
    \end{itemize}
    \item \faServer{}\ HPC cluster (\textnormal{Slurm \& Torque)}
    \item \faGit*{}\ Git, GitHub \& git-annex \textnormal{(Datalad)}
    \item \faLaptop{}\ Chat-GPT
    \item \faBook{}\ Codebooks \textnormal{(Quarto, Jupyter, etc.)}
    \item \faPencil*{}\ MS Office, Markdown \& LaTeX
\end{itemize}
%\vspace{0.5em}
\divider{}
\bfseries\textcolor{emphasis}{Languages}
\vspace{1ex}
\begin{itemize}
    \item Deutsch \textnormal{(Muttersprache)}
    \item English \textnormal{(C1/C2 + 6 Jahre Berufssprache)}
\end{itemize}
\vspace{0.5ex}

\medskip
\cvsection{Bildung}
\cvedu{Promotion (Dr. rer. nat.)}{Max-Planck-Institut für Bildungsforschung \& Freie Universität, Berlin}{11/2017~--~12/2022}{}
\cvedu{M.Sc.\ Cognitive Neuroscience}{Ruhr-Universität, Bochum}{10/2014~--~03/2017}{Grade: 1.2 (very good)}
\cvedu{B.Sc.\ Psychologie}{Ruhr-Universität, Bochum}{10/2011~--~07/2014}{Grade: 1.3 (very good)}

%\cvsection{Strengths}
%\cvtag{\faFlagCheckered{}Driven}
%\cvtag{\faListOl{}Organized}
%\cvtag{\faPalette{}Creative}
%\cvtag{\faUserPlus{}Team Player}
%\cvtag{\faSpa{}Conscientious}

%% Yeah I didn't spend too much time making all the
%% spacing consistent... sorry. Use \smallskip, \medskip,
%% \bigskip, \vspace etc to make adjustments.
\bigskip

\end{paracol}
\vspace*{-1.2em}
\cvsection{References}

% Three column layout
% Set equal spacing among the three columns
%\columnratio{0.333}
% Add vertical lines as dividers
%\setlength{\columnseprule}{0.5pt}
% Start columns
%\begin{paracol}{3}
  % Actvate first column
  %\switchcolumn[0]
  % Add reference
  %\cvref{Dr. Place Holder}{Lead of Placeholding\par\hspace{0.5em}Places Holding Inc.\par\hspace{0.5em}Holdings Place, PH}{\href{mailto:placeholder@mail.com}{placeholder@mail.com}}
  %\switchcolumn[1]
  %\cvref{Dr. Place Holder}{Lead of Placeholding\par\hspace{0.5em}Places Holding Inc.\par\hspace{0.5em}Holdings Place, PH}{\href{mailto:placeholder@mail.com}{placeholder@mail.com}}
  %\switchcolumn[2]
  %\cvref{Dr. Place Holder}{Lead of Placeholding\par\hspace{0.5em}Places Holding Inc.\par\hspace{0.5em}Holdings Place, PH}{\href{mailto:placeholder@mail.com}{placeholder@mail.com}}
% End column layout
%\end{paracol}

\vspace{-20pt}
\begin{center}
    \begin{minipage}[t][10pt]{0.31\textwidth}
        \cvref{Prof.~Dr.~Nicolas Schuck}{Research Group Leader\par\hspace{0.5em}Universität Hamburg\par\hspace{0.5em}Hamburg}{\href{mailto:nicolas.schuck@uni-hamburg.de}{nicolas.schuck@uni-hamburg.de}}
    \end{minipage}
    \hfill
    \begin{minipage}[t][10pt]{0.25\textwidth}
        \cvref{Dr. Lorena Deuker}{Analytics Group Leader\par\hspace{0.5em}Lidl International\par\hspace{0.5em}Neckarsulm}{\href{mailto:lorenadeuker@gmail.com}{lorenadeuker@gmail.com}}
    \end{minipage}
    \hfill
    \begin{minipage}[t][10pt]{0.29\textwidth}
        \cvref{Dr. Ondrej Zika}{Senior Research Scientist\par\hspace{0.5em}MPI für Bildungsforschung\par\hspace{0.5em}Berlin}{\href{mailto:zika@mpib-berlin.mpg.de}{zika@mpib-berlin.mpg.de}}
    \end{minipage}
\end{center}


\end{document}